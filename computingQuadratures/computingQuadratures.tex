\documentclass{ximera}

\preambleinput{../preamble.tex}

\prerequisites{geometry}
\outcomes{quadratures}

\title{Computing quadratures}

\begin{document}
\begin{abstract}
In this activity we will compute some basic quadratures.
\end{abstract}
\maketitle


When computing a quadrature of a shape in the method of the ancient
Greeks, one needs to produce a line segment whose length gives the
side of a square of equal area to the original shape.


\begin{question}
Consider the figure below. Explain how one could construct it and
what segment $x$ represents.
\begin{image}
\begin{tikzpicture}[geometryDiagrams]
\draw[thin] (2,0) arc (0:180:2cm);
\draw[thin] (-2,0)--(1,0);
\draw[thin] (1,0)--(2,0);
\draw[decoration={brace,mirror,raise=.2cm},decorate,thin] (-1.9,0)--(.9,0);
\draw[decoration={brace,mirror,raise=.2cm},decorate,thin] (1.1,0)--(1.9,0);
\draw (1,0)--({2*cos(60)},{2*sin(60)});
\node at (.8,{sin(60)}) {$x$};
\node at (-.5,-.5) {$n$};
\node at (1.5,-.5) {$1$};
\end{tikzpicture}
\end{image}
\end{question}


\begin{question}
How do you compute the quadrature of an $a\times b$ rectangle?
\end{question}

\begin{question}
How do you compute the quadrature of a triangle?
\end{question}


\begin{question}
How do you compute the quadrature of several squares?
\end{question}


\begin{question}
How do you compute the quadrature of a polygon?
\end{question}
\end{document}
