\documentclass{ximera}

\title{Summary}

\begin{document}
\begin{abstract}
Here we see a summary of our course thus far. 
\end{abstract}
\maketitle

Let's review each of the activities we've completed.

\begin{description}
\item[What is a number?] 20000 BCE---present. Numbers are
  identified/concepualized by their relations to other numbers.
\item[Measuring with fractions] 2050 BCE--500
  CE. The concept of a number is stongly connected to
  a number's use. 
\item[The method of false position] 1650 BCE--1850
  CE. The method of false position is proceedural
  algorithm, that reduces linear equations to a rote, algorithmetic
  proceedure. This proceedure is based on the notion of slope.
\item[Babylonian numbers] 3100 BCE-540 BCE. The
  Babylonian system is the first place-value system.
\item[Rational numbers and similarity] 500
  BCE--300 BCE. Here we explore how mathematics might look to a
  culture who believes that every number is rational.
\item[Pythagorean means] 500 BCE--600 CE. This is an overview of the
  three means. Given a set of (equivalent) units that are recripricals
  of one another, conversion from one unit to the other may change
  which mean we are considering.
\item[Computing quadritures] 500 BCE--600 CE. Here we think about
  area, units, and try to put ourselves in the mindset of the ancient
  Greeks as they wrestled with these concepts.
\item[Squaring the circle with lunes] 500 BCE--600 CE. Here we follow
  an intriguing plan to that seems like it will allow us to compute
  the quariture of the circle.
\item[Euclid's Elements] 500 BCE--1850 CE. Here we examine Euclid's
  elements and prove basic theorems from geometry.
\item[Triangles on a cone] 500 BCE--present. Here we start to
  appreciate some of Euclid's axioms, as we find ourselves working
  with a geometry where some do not hold.
\item[The Pythagorean Theorem] 500 BCE--present. Here we give several
  proofs of the most famous theorem of all.
\item[The unique factorization theorem] 500 BCE--present. Here we
  investigate Euclid's proof of the unique factorization theorem of
  the integers, and we see a simple system where it fails.
\item[Heron's formula] 500 BCE--present. Here we investigate Heron's proof of Heron's formula.
\item[Solving equations] 2500 BCE--present.
\end{description}


\end{document}
