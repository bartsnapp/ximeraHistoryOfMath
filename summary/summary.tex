\documentclass{ximera}

\title{Summary}

\begin{document}
\begin{abstract}
Here we see a summary of our course thus far. 
\end{abstract}
\maketitle

Let's review each of the activities we've completed. Here we have a
list with approximate dates. Fill in the primary content of each
activity.

\begin{description}
\item[What is a number?] 20000 BCE.
\vspace{1in}\item[Measuring with fractions] 2050 BCE.
\vspace{1in}\item[The method of false position] 1650 BCE.
\vspace{1in}\item[Babylonian numbers] 3100 BCE.
\vspace{1in}\item[Rational numbers and similarity] 500 BCE.
\vspace{1in}\item[Pythagorean means] 500 BCE.
\vspace{1in}\item[Computing quadratures] 500 BCE.
\vspace{1in}\item[Squaring the circle with lunes] 210 CE.
\vspace{1in}\item[Euclid's Elements] 300 BCE.
\vspace{1in}\item[Triangles on a cone] 300 BCE.
\vspace{1in}\item[The Pythagorean Theorem] 300 BCE.
\vspace{1in}\item[The unique factorization theorem] 300 BCE.
\vspace{1in}\item[Heron's formula] 0 CE.
\vspace{1in}\item[Solving equations] 2500 BCE, 1500 CE.
\vspace{1in}\item[Complex numbers from different angles] 1700 CE.
\vspace{1in}\item[De Moivre saves the day!] 1700 CE.
\vspace{1in}\item[The binomial theorem and $\pi$] 1700 CE.
\vspace{1in}\item[Newton and Kepler $\pi$] 1700 CE.
\vspace{1in}\item[Leibniz and series] 1700 CE.
\vspace{1in}\item[Bernoulli, Euler, and series] 1750 CE.
\vspace{1in}\item[Cantor can!] 1850 CE.
\vspace{1in}\item[The foundations of geometry] 1890 CE.
\vspace{1in}\item[City geometry] 1890 CE.
\vspace{1in}\item[Limits of axioms] 1930 CE.
\vspace{1in}
\end{description}


\end{document}
