\documentclass{ximera}

\preambleinput{../preamble.tex}


\title{Cantor Can!}
\begin{document}
\begin{abstract}
In this activity we look at Cantor's diagonal argument.
\end{abstract}
\maketitle

It took until the 1700's to get algebra and number systems in place in
a workable way.  But there was still trouble understanding what
infinity was.  Was the set of counting numbers really infinite, or was
it only as big as the highest number that anyone had ever counted, or
as big as the number of atoms in the universe, or\dots?  But even if
the set of counting numbers was infinite, then the set of real numbers
was also infinite.  But then again, were they the same infinity?  Some
math grad student in Germany around 1850 shocked the math world by
saying `no.'


\begin{question}
Here is a table of rational numbers:
\[
{\renewcommand{\arraystretch}{3}
\begin{array}{|c|c|c|c|c|c|c|c|c|c|c|c|c|}\hline
\cdots & -5 & -4 & -3 & -2 & -1 & 0 & 1 & 2 & 3 & 4 & 5 & \cdots\\\hline
\cdots &\frac{-5}{2} & & \frac{-3}{2} &  & \frac{-1}{2} &   & \frac{1}{2} &  & \frac{3}{2} &  & \frac{5}{2} & \cdots\\\hline
\cdots & \frac{-5}{3} & \frac{-4}{3} & & \frac{-2}{3} & \frac{-1}{3} & & \frac{1}{3} & \frac{2}{3} & & \frac{4}{3} & \frac{5}{3} & \cdots\\\hline
\cdots & \frac{-5}{4} & & \frac{-3}{4} & & \frac{-1}{4} & & \frac{1}{4} & & \frac{3}{4} & & \frac{5}{4} & \cdots\\\hline
\cdots &  & \frac{-4}{5} & \frac{-3}{5} & \frac{-2}{5} & \frac{-1}{5} & & \frac{1}{5} & \frac{2}{5} & \frac{3}{5} & \frac{4}{5} &  & \cdots\\\hline
       & \vdots & \vdots & \vdots & \vdots & \vdots & \vdots & \vdots & \vdots & \vdots & \vdots & \vdots & \\\hline
\end{array}}
\]
\begin{enumerate}
\item What does the 12th row of the table look like? 
\item Name three different rational numbers. Will they (eventually) appear on the table?
\item Will every rational number eventually appear in the table above?
\item Can you figure out how to ``enumerate'' the rationals?
\end{enumerate}
\end{question}


\begin{question}
The question: Are the set of counting numbers and the set of real
numbers between $0$ and $1$ the same size?

Cantor's answer: Suppose they were, then you could make a one-to-one,
onto match-up:
\begin{align*}
1 &:0\,.\,2\,2\,3\,4\,3\,7\,9\,8\,7\,8\,4\,\dots\\
2 &:0\,.\,8\,5\,9\,8\,4\,7\,5\,9\,3\,4\,8\,\dots\\
3 &:0\,.\,1\,1\,2\,9\,0\,2\,9\,3\,9\,8\,0\,\dots\\
4 &:0\,.\,0\,3\,4\,3\,2\,3\,4\,0\,5\,6\,3\,\dots\\
5 &:0\,.\,9\,3\,9\,2\,8\,4\,9\,8\,2\,3\,9\,\dots\\
6 &:0\,.\,7\,9\,7\,8\,8\,9\,3\,7\,8\,3\,3\,\dots\\
  &\;\vdots
\end{align*}
So, you think you did it, eh?  I will find a real number between zero
and one that is not on your list.  How will I do it?
\end{question}

\begin{question}
Explain why the same argument does \textit{not} show that the
rationals cannot be enumerated.
\end{question}


\begin{question}
Given any \textit{finite} set $S$, can you prove that the power set of $S$ has a
larger cardinality?
\end{question}

\begin{question}
Given any set $S$, can you prove that the power set of $S$ has a
larger cardinality? Hint: Attempt to ``count'' $\P(S)$ with $A_x$ where
$x\in S$. Then consider $A \subset \P(S)$ where
\[
A = \{x : x \in S\text{ and } x\not\in A_x\}.
\]
\end{question}



\begin{exploration}
Is the cardinality of $\R$ equal to the cardinality of $\P(\Q)$?
\end{exploration}

\end{document}
