\documentclass{ximera}

\preambleinput{../preamble}


\prerequisites{algebra}
\outcomes{egyptianFractions,placeValue}

\title{Measuring with fractions}
\begin{document}
\begin{abstract}
In this activity we will explore Egyptian fractions.
\end{abstract}
\maketitle

Here is a basic question for you: 

\begin{question}
What is the smallest number of weights needed to produce every
integer-valued mass from $0$ grams to say $n$ grams? Explain your
reasoning.
\end{question}

Now suppose that you want to measure all fractions of a given unit
with unit fractions. 

\begin{exercise}
Express 
\[
\frac{2}{5}, \qquad \frac{3}{5}, \qquad \frac{4}{5} 
\]
as the sum of unit fractions. 
\end{exercise}

Why would somebody work with such beasts? I'm not sure, but I have two
guesses. 

First guess: Egyptians weren't using fractions for fun, they wanted to do
real measurements. One could imagine a market where to measure weights
one has a balance. If you had a fixed set of weights that allowed you
to measure many different types of fractional weights, the
decomposition of a fraction into unit sums would be of interest.

Second guess: It has to do with distributing in a real world setting.

\begin{question}
Suppose you have 7 loafs of bread that you wish to share among 8
people. How much of a loaf of bread do you need to give each person?
How \textbf{exactly} do you distribute the bread?
\end{question}


\begin{exploration}
Express 
\[
\frac{2}{7}, \qquad \frac{3}{7}, \qquad \frac{4}{7}, \qquad\frac{5}{7}, \qquad\frac{6}{7} 
\]
as the sum of unit fractions. Can you give a general method?
\end{exploration}
\end{document}
