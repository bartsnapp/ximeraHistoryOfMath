\documentclass{ximera}

\input{../preamble.tex}

\prerequisites{algebra}
\outcomes{conceptOfNumber}

\title{What is a number?}
\begin{document}
\begin{abstract}
In this activity we think about what it means to be a number.
\end{abstract}
\maketitle

\begin{question}
What is a number? List out some qualities that you think a number has.
\end{question}

\begin{question}
How do you know when something is not a number?
\end{question}

\begin{question}
What are different ways we could represent numbers?
\end{question}

\begin{question}
What if we used the following system:
\[
1 = a, \quad 2 = aa, \quad 3 = aaa, \quad\text{and so on.}
\]
What does the word \textit{concatenation} mean? How does it apply to
this system?  How would we add? How would we multiply? How could we
represent negative numbers?
\end{question}


\begin{question}
Once Oscar wondered what the number $\pi$ was. So he typed it into his
calculator and found:
\[
\pi = 3.1415926\dots
\]
Oscar then exclaimed, ``Ah now I know what number $\pi$ is.'' Can you
explain Oscar's thoughts on numbers?
\end{question}


\begin{exercise}
What are the 7 basic ancient Egyptian numerical symbols and what do
they mean?
\end{exercise}


\begin{question}
Is this ancient Egyptian system a place-value system or a
concatenation system? Explain your reasoning.
\end{question}


\begin{question}
Suppose you wanted to write out all the numbers from 1 to 1000000. How
many distinct symbols would you need if using the ancient Egyptian
system? How many distinct symbols would you need if using our system?
How many symbols would the longest string require? 
\end{question}

\begin{question}
Is there a simple rule for multiplying by 10 in the ancient Egyptian system?
\end{question}
\end{document}
