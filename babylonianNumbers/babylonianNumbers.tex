\documentclass[handout,newpage]{ximera}

%\prerequisites{algebra}
%\outcomes{placeValue,squareRoots}
\input{../preamble.tex}


\title{Babylonian numbers} 


\begin{document}
\begin{abstract}In this activity we explore the number system of the ancient
  Babylonians.
\end{abstract} 
\maketitle


The ancient Babylonians used cuneiform characters to write their
numbers.

\begin{exercise}
What are the 2 basic ancient Babylonian numerical symbols and what do
they mean?
\end{exercise}


\begin{exercise}
Express the numbers 
\[
1, \qquad 7,\qquad 13,\qquad 53,\qquad 101 
\]
as the ancient Babylonians would. 
\end{exercise}


\begin{question}
Express 0 as the ancient Babylonians would. 
\end{question}


\begin{question}
Count from 58 to 62 using the ancient Babylonian symbols. 
\end{question}



\begin{exploration}
Express the numbers 
\[
11,\qquad 660,\qquad 39600 
\]
as the ancient Babylonians would. 
\end{exploration}


\begin{exploration}
Discuss the limitations of the Babylonian system. Then debate whether
these so-called limitations were actually limitations at all.
\end{exploration}

\begin{exploration}
Is the Babylonian system more of a place-value system or a
concatenation system?
\end{exploration}



\begin{question}%% note difficult problem from math through ages
Express the numbers 
\[
\frac{5}{6}, \qquad \frac{1}{20},\qquad \frac{1}{100}
\]
in sexagesimal notation.
\end{question}



\begin{question}
Here is a computation of how ancient Babylonians approximated square
roots---without any explanation!
\begin{enumerate}
\item $\sqrt{26}> 5$
\item $5 \cdot \frac{26}{5} = 26$
\item $\frac{26}{5}> \sqrt{26}$
\item $\sqrt{26}\approx \frac{5+\frac{26}{5}}{2}$
\end{enumerate}
Explain the algorithm used and give another example to show you know how it
is done.
\end{question}

\begin{exploration}
When is this approximation a good approximation?
\end{exploration}

\end{document}
