\documentclass{ximera}

\preambleinput{../preamble.tex}
\title{Rational numbers and similarity}

\begin{document}
\begin{abstract}
In this activity we play a game of ``what if'' and see a reason that
  the ancient Greeks might have wanted every number to be rational.
\end{abstract}  
\maketitle

\begin{exploration}
Think about plain old plane geometry. What are some theorems that you
would want to be true?
\end{exploration}



\begin{question}
What are the basic theorems involving similar triangles?
\end{question}

OK---now we are going to do something very strange. Let's suppose that
every number is rational. In essence, let's put ourselves into the
mindset of the ancient Greeks, \textbf{before} they knew that
irrational numbers existed.

\begin{exploration}
Suppose that you have two triangles whose angles are congruent. Can
you make a fairly simple argument, using the fact that the sides are
rational numbers, that shows that the sides are proportional? Hint:
You may need to use ASA.
\end{exploration}


\begin{exploration}
Suppose that you have two triangles whose sides are proportional. Can
you make a fairly simple argument, using the fact that the sides are
rational numbers, that shows that the angles are congruent? Hint: You
may need to use SSS.
\end{exploration}
\end{document}
